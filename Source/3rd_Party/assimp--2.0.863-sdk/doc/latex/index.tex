\hypertarget{index_intro}{}\section{Introduction}\label{index_intro}
A\+S\+S\+I\+M\+P is a library to load and process geometric scenes from various data formats. It is tailored at typical game scenarios by supporting a node hierarchy, static or skinned meshes, materials, bone animations and potential texture data. The library is {\itshape not} designed for speed, it is primarily useful for importing assets from various sources once and storing it in a engine-\/specific format for easy and fast every-\/day-\/loading. A\+S\+S\+I\+M\+P is also able to apply various post processing steps to the imported data such as conversion to indexed meshes, calculation of normals or tangents/bitangents or conversion from right-\/handed to left-\/handed coordinate systems.

A\+S\+S\+I\+M\+P currently supports the following file formats (note that some loaders lack some features of their formats because some file formats contain data not supported by A\+S\+S\+I\+M\+P, some stuff would require so much conversion work that it has not been implemented yet and some (most ...) formats lack proper specifications)\+: 

 ~\newline
{\ttfamily  {\bfseries Collada} ( {\itshape $\ast$.dae;$\ast$.xml} )~\newline
 {\bfseries Blender} ( {\itshape $\ast$.blend} ) \textsuperscript{3}~\newline
 {\bfseries Biovision B\+V\+H } ( {\itshape $\ast$.bvh} ) ~\newline
 {\bfseries 3\+D Studio Max 3\+D\+S} ( {\itshape $\ast$.3ds} ) ~\newline
 {\bfseries 3\+D Studio Max A\+S\+E} ( {\itshape $\ast$.ase} ) ~\newline
 {\bfseries Wavefront Object} ( {\itshape $\ast$.obj} ) ~\newline
 {\bfseries Stanford Polygon Library} ( {\itshape $\ast$.ply} ) ~\newline
 {\bfseries Auto\+C\+A\+D D\+X\+F} ( {\itshape $\ast$.dxf} ) \textsuperscript{2}~\newline
 {\bfseries Neutral File Format} ( {\itshape $\ast$.nff} ) ~\newline
 {\bfseries Sense8 World\+Toolkit} ( {\itshape $\ast$.nff} ) ~\newline
 {\bfseries Valve Model} ( {\itshape $\ast$.smd,$\ast$.vta} ) \textsuperscript{3} ~\newline
 {\bfseries Quake I} ( {\itshape $\ast$.mdl} ) ~\newline
 {\bfseries Quake I\+I} ( {\itshape $\ast$.md2} ) ~\newline
 {\bfseries Quake I\+I\+I} ( {\itshape $\ast$.md3} ) ~\newline
 {\bfseries Quake 3 B\+S\+P} ( {\itshape $\ast$.pk3} ) \textsuperscript{1} ~\newline
 {\bfseries Rt\+C\+W} ( {\itshape $\ast$.mdc} )~\newline
 {\bfseries Doom 3} ( {\itshape $\ast$.md5mesh;$\ast$.md5anim;$\ast$.md5camera} ) ~\newline
 {\bfseries Direct\+X X } ( {\itshape $\ast$.x} ). ~\newline
 {\bfseries Quick3\+D } ( {\itshape $\ast$.q3o;{\itshape q3s} ). ~\newline
 {\bfseries Raw Triangles } ( {\itshape }.raw} ). ~\newline
 {\bfseries A\+C3\+D } ( {\itshape $\ast$.ac} ). ~\newline
 {\bfseries Stereolithography } ( {\itshape $\ast$.stl} ). ~\newline
 {\bfseries Autodesk D\+X\+F } ( {\itshape $\ast$.dxf} ). ~\newline
 {\bfseries Irrlicht Mesh } ( {\itshape $\ast$.irrmesh;$\ast$.xml} ). ~\newline
 {\bfseries Irrlicht Scene } ( {\itshape $\ast$.irr;$\ast$.xml} ). ~\newline
 {\bfseries Object File Format } ( {\itshape $\ast$.off} ). ~\newline
 {\bfseries Terragen Terrain } ( {\itshape $\ast$.ter} ) ~\newline
 {\bfseries 3\+D Game\+Studio Model } ( {\itshape $\ast$.mdl} ) ~\newline
 {\bfseries 3\+D Game\+Studio Terrain} ( {\itshape $\ast$.hmp} )~\newline
 {\bfseries Ogre} ({\itshape $\ast$.mesh.\+xml, $\ast$.skeleton.\+xml, $\ast$.material})\textsuperscript{3} ~\newline
 {\bfseries Milkshape 3\+D} ( {\itshape $\ast$.ms3d} )~\newline
 {\bfseries Light\+Wave Model} ( {\itshape $\ast$.lwo} )~\newline
 {\bfseries Light\+Wave Scene} ( {\itshape $\ast$.lws} )~\newline
 {\bfseries Modo Model} ( {\itshape $\ast$.lxo} )~\newline
 {\bfseries Character\+Studio Motion} ( {\itshape $\ast$.csm} )~\newline
 {\bfseries Stanford Ply} ( {\itshape $\ast$.ply} )~\newline
 {\bfseries True\+Space} ( {\itshape $\ast$.cob, $\ast$.scn} )\textsuperscript{2}~\newline
~\newline
 } See the \hyperlink{importer_notes}{Importer Notes Page } for informations, what a specific importer can do and what not. Note that although this paper claims to be the official documentation, \href{http://assimp.sourceforge.net/main_features_formats.html}{\tt http\+://assimp.\+sourceforge.\+net/main\+\_\+features\+\_\+formats.\+html} ~\newline
is usually the most up-\/to-\/date list of file formats supported by the library. ~\newline


\textsuperscript{1}\+: Experimental loaders~\newline
 \textsuperscript{2}\+: Indicates very limited support -\/ many of the format\textquotesingle{}s features don\textquotesingle{}t map to Assimp\textquotesingle{}s data structures.~\newline
 \textsuperscript{3}\+: These formats support animations, but A\+S\+S\+I\+M\+P doesn\textquotesingle{}t yet support them (or they\textquotesingle{}re buggy)~\newline
 ~\newline
 



A\+S\+S\+I\+M\+P is independent of the Operating System by nature, providing a C++ interface for easy integration with game engines and a C interface to allow bindings to other programming languages. At the moment the library runs on any little-\/endian platform including X86/\+Windows/\+Linux/\+Mac and X64/\+Windows/\+Linux/\+Mac. Special attention was paid to keep the library as free as possible from dependencies.

Big endian systems such as P\+P\+C-\/\+Macs or P\+P\+C-\/\+Linux systems are not officially supported at the moment. However, most formats handle the required endian conversion correctly, so large parts of the library should work.

The A\+S\+S\+I\+M\+P linker library and viewer application are provided under the B\+S\+D 3-\/clause license. This basically means that you are free to use it in open-\/ or closed-\/source projects, for commercial or non-\/commercial purposes as you like as long as you retain the license informations and take own responsibility for what you do with it. For details see the L\+I\+C\+E\+N\+S\+E file.

You can find test models for almost all formats in the $<$assimp\+\_\+root$>$/test/models directory. Beware, they\textquotesingle{}re {\itshape free}, but not all of them are {\itshape open-\/source}. If there\textquotesingle{}s an accompagning \textquotesingle{}$<$file$>$.txt\textquotesingle{} file don\textquotesingle{}t forget to read it.\hypertarget{index_main_install}{}\section{Installation}\label{index_main_install}
A\+S\+S\+I\+M\+P can be used in two ways\+: linking against the pre-\/built libraries or building the library on your own. The former option is the easiest, but the A\+S\+S\+I\+M\+P distribution contains pre-\/built libraries only for Visual C++ 2005 and 2008. For other compilers you\textquotesingle{}ll have to build A\+S\+S\+I\+M\+P for yourself. Which is hopefully as hassle-\/free as the other way, but needs a bit more work. Both ways are described at the \hyperlink{install}{Installation page. }\hypertarget{index_main_usage}{}\section{Usage}\label{index_main_usage}
When you\textquotesingle{}re done integrating the library into your I\+D\+E / project, you can now start using it. There are two separate interfaces by which you can access the library\+: a C++ interface and a C interface using flat functions. While the former is easier to handle, the latter also forms a point where other programming languages can connect to. Upto the moment, though, there are no bindings for any other language provided. Have a look at the \hyperlink{usage}{Usage page } for a detailed explanation and code examples.\hypertarget{index_main_data}{}\section{Data Structures}\label{index_main_data}
When the importer successfully completed its job, the imported data is returned in an ai\+Scene structure. This is the root point from where you can access all the various data types that a scene/model file can possibly contain. The \hyperlink{data}{Data Structures page } describes how to interpret this data.\hypertarget{index_ext}{}\section{Extending the library}\label{index_ext}
There are many 3d file formats in the world, and we\textquotesingle{}re happy to support as many as possible. If you need support for a particular file format, why not implement it yourself and add it to the library? Writing importer plugins for A\+S\+S\+I\+M\+P is considerably easy, as the whole postprocessing infrastructure is available and does much of the work for you. See the \hyperlink{extend}{Extending the library } page for more information.\hypertarget{index_main_viewer}{}\section{The Viewer}\label{index_main_viewer}
The A\+S\+S\+I\+M\+P viewer is a standalone Windows/\+Direct\+X application that was developed along with the library. It is very useful for quickly examining the contents of a scene file and test the suitability of its contents for realtime rendering. The viewer offers a lot of additional features to view, interact with or export bits of the data. See the \hyperlink{viewer}{Viewer page } for a detailed description of its capabilities.\hypertarget{index_main_support}{}\section{Support \& Feedback}\label{index_main_support}
If you have any questions/comments/suggestions/bug reports you\textquotesingle{}re welcome to post them in our \href{https://sourceforge.net/forum/forum.php?forum_id=817653}{\tt forums}. Alternatively there\textquotesingle{}s a mailing list, \href{https://sourceforge.net/mailarchive/forum.php?forum_name=assimp-discussions}{\tt assimp-\/discussions}. 