\hypertarget{perf_perf_overview}{}\section{Overview}\label{perf_perf_overview}
This page discusses Assimps general performance and some ways to finetune and profile it. You will see that an intelligent choice of postprocessing steps is essential for quick loading.\hypertarget{perf_perf_profile}{}\section{Profiling}\label{perf_perf_profile}
Assimp has builtin support for basic profiling and reporting. To turn it on, set the {\ttfamily G\+L\+O\+B\+\_\+\+M\+E\+A\+S\+U\+R\+E\+\_\+\+T\+I\+M\+E} configuration switch to {\ttfamily true} (nonzero). Results are dumped to the logfile, so you need to setup an appropriate logger implementation with at least one output stream first. See the \hyperlink{usage_logging}{Logging Page } for the details.

A sample report looks like this (some unrelated log messages omitted, grouped entries for clarity)\+:

\begin{DoxyVerb}Debug, T5488: START `total`
Info,  T5488: Found a matching importer for this file format


Debug, T5488: START `import`
Info,  T5488: BlendModifier: Applied the `Subdivision` modifier to `OBMonkey`
Debug, T5488: END   `import`, dt= 3.516 s


Debug, T5488: START `preprocess`
Debug, T5488: END   `preprocess`, dt= 0.001 s
Info,  T5488: Entering post processing pipeline


Debug, T5488: START `postprocess`
Debug, T5488: RemoveRedundantMatsProcess begin
Debug, T5488: RemoveRedundantMatsProcess finished 
Debug, T5488: END   `postprocess`, dt= 0.001 s


Debug, T5488: START `postprocess`
Debug, T5488: TriangulateProcess begin
Info,  T5488: TriangulateProcess finished. All polygons have been triangulated.
Debug, T5488: END   `postprocess`, dt= 3.415 s


Debug, T5488: START `postprocess`
Debug, T5488: SortByPTypeProcess begin
Info,  T5488: Points: 0, Lines: 0, Triangles: 1, Polygons: 0 (Meshes, X = removed)
Debug, T5488: SortByPTypeProcess finished

Debug, T5488: START `postprocess`
Debug, T5488: JoinVerticesProcess begin
Debug, T5488: Mesh 0 (unnamed) | Verts in: 503808 out: 126345 | ~74.922
Info,  T5488: JoinVerticesProcess finished | Verts in: 503808 out: 126345 | ~74.9
Debug, T5488: END   `postprocess`, dt= 2.052 s

Debug, T5488: START `postprocess`
Debug, T5488: FlipWindingOrderProcess begin
Debug, T5488: FlipWindingOrderProcess finished
Debug, T5488: END   `postprocess`, dt= 0.006 s


Debug, T5488: START `postprocess`
Debug, T5488: LimitBoneWeightsProcess begin
Debug, T5488: LimitBoneWeightsProcess end
Debug, T5488: END   `postprocess`, dt= 0.001 s


Debug, T5488: START `postprocess`
Debug, T5488: ImproveCacheLocalityProcess begin
Debug, T5488: Mesh 0 | ACMR in: 0.851622 out: 0.718139 | ~15.7
Info,  T5488: Cache relevant are 1 meshes (251904 faces). Average output ACMR is 0.718139
Debug, T5488: ImproveCacheLocalityProcess finished. 
Debug, T5488: END   `postprocess`, dt= 1.903 s


Info,  T5488: Leaving post processing pipeline
Debug, T5488: END   `total`, dt= 11.269 s
\end{DoxyVerb}


So, only one fourth of the total import time was used for the actual model import, while the rest of the time was consumed by the \#ai\+Process\+\_\+\+Triangulate, \#ai\+Process\+\_\+\+Join\+Identical\+Vertices and \#ai\+Process\+\_\+\+Improve\+Cache\+Locality postprocessing steps. It is therefore not a good idea to specify {\itshape all} postprocessing flags just because they sound so nice. 